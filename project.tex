\documentclass{article}
% Comment the following line to NOT allow the usage of umlauts
\usepackage{fullpage}
\usepackage[utf8]{inputenc}
\usepackage{amsmath}
\usepackage{amsfonts}
\usepackage{amssymb}
\renewcommand{\baselinestretch}{1.5}
\newcommand\tab[1][1cm]{\hspace*{1}}
\pagenumbering{arabic}
\usepackage{comment }
\usepackage{multicol}
\usepackage{geometry}
%\usepackage{sectsty}
%\usepackage{scrextend}
%\changefontsizes[40pt]{25pt}
%\usepackage[english]{babel}


\begin{document}
	\begin{center}
	\Huge\textbf{Ajibola Segun Emmanuel} \\
	\vspace{5mm} %5mm vertical space
	\Huge\textbf{180551012} \\
	\vspace{5mm} %5mm vertical space
	\Huge\textbf{B.Sc Mathematics} \\ 
	\end{center}
	\vspace{50mm} %20mm vertical space
	\begin{center}
	\Large\textbf{PROJECT TOPIC:} \\
	\vspace{5mm} %5mm vertical space
	\Huge\textbf{\Huge OPTIMIZATION FOR TRANSPORTATION PROBLEMS FOR THREE NIGERIAN TRANSPORTING COMPANIES}\par
	\end{center}

\newpage
\begin{center}
	\subsection{Certification}
\end{center}

I certify that this work by AJIBOLA SEGUN EMMANUEL with matriculation number 180551012 under my supervision in the Department of Mathematics, Lagos State University, Ojo, Lagos.

\vspace{30mm}

\begin{multicols}{2}

\noindent
Mr Oni \\ 
(Supervisor) \\ 
Department of Mathematics \\ 
Lagos State University \\ 
Ojo, Lagos. \\ 

\noindent
..................... \\
Signature \\
..................... \\
Date \\

\noindent
Dr Oyebo Y. T. \\
Ag. Head \\
Department of Mathematics \\
Lagos State University \\
Ojo, Lagos. \\

\noindent
..................... \\
Signature \\
..................... \\
Date \\

\end{multicols}

\newpage

\begin{center}
	\subsection{Dedication}
\end{center}


I dedicate this seminar work to God Almighty for his protection over my course of study, and also to my family, the family of Mr and Mrs Ajibola, my relatives and friends. Thanks for all the support.

\newpage

\begin{center}
	\subsection{Acknowledgement}
\end{center}

I am grateful to God for giving me the ability to work on this project. I also thank my supervisor, Mr Oni, for his guidance along this project work.

My gratitude also goes to my parents, relatives, coursemates, and friends for their sincere love and support during my time in school. May God bless you all.

\newpage

\begin{center}
	\subsection{Abstract}
\end{center}


Operation Research, also shorten as OR, is a field that deals with the development and application of analytical methods to improve decision making. 

It is used to solve complex problems ranging from business to art through mathematical analysis. Its technique can be applied in various fields of agriculture, transportation and business management.

Operation reaserch's method to optimize conditions can affect humans decision and make them make life-saving decisions.

\newpage

\subsection{Table of Contents}
\tableofcontents

\newpage

\begin{center}
\section{INTRODUCTION}
\end{center}
Bjarke Bundgaard Ingels, a Danish architect, founder and creative partner of Bjarke Ingels Group said "In Copenhagen, there's a long-term commitment to creating a well-functioning pedestrian city where all forms of movement - pedestrian, bicycles, cars, public transportation - are accommodated with equal priority". As at 2022, Elon Musk, the richest man on earth said "If we drive down the cost of transportation in space, we can do great things". This two great quotes from experienced people in different fields shows the importance of transportation in our daily lives and businesses.

During the administration of President Olusegun Obasanjo in Nigeria, the National Poverty Eradication Programme (NAPEP) program was created in 2001 by the Nigerian government aiming to reduce poverty and provide economic empowerment to the people of the country. This prompted the high use of the popular tricyle known as Keke Maruwa and it was soon popularly known as Keke Napep across the nation. This notion was popular and can be attributed to the widespread National Poverty Eradication Programme NAPEP, which was inaugurated under the current government.

Without any doubt, this short story clearly shows that transportation is one of most important functions in any country, developing or developed countries. It is one of the core activity necessary for the advancement of human life and the growth in industrial sector.

In Nigeria, the length of distance travelled is over two hundred thousand kilometres in road, rail, water ways, commercial harbours and airports sector.

\newpage

\begin{center}
\subsection{Statement of Problem}
\end{center}
With the high kilometres travelled in Nigeria, this brings risks and cost problems for industries sending goods from one place in and out of the country.

As good and important the transport system of a state can be, it's disadvantages poses a high cost. Millions can be lost in a day and in most bad cases, human lives can be lost.

Poor transport system, design and decisions increases the damages in movement of materials from one place to another to conduct business activities.

According to Logistics Update Africa, Nigeria's Federal Road Safety Corps disclosed that Nigeria lost N39 billion to trailer and tanker crashes in 2018, involving about 650 vehicles.

\newpage

\begin{center}
	\subsection{Scope of Study}
\end{center}
This project will highlight the transportation problem in the sector. It will use an operation research called transportation problem to analyse a transport system and enumerates ways that cost of bad transportation problems can be reduced, thus increasing income.

Linear programming will be used in this project, the type of the programming used is called transportation problem. It is a type of linear programming where the objective helps to minimize transportation cost of goods from a place to different locations.

Three (3) pure water factories in Nigeria will be used as examples, the transportation details will be stated and the transportation optimisation system will be used to reduce cost of transportation to get the maximum profits.

\newpage

\begin{center}
\subsection{Objectives of Study}
\end{center}

The objectives of this project is stated below: \newline
\begin{enumerate}
\item Enumerate the importance of optimized transport system
\item Review the Transportation Problem of Linear Programming
\item Use the application of Linear Programming to optimize the transportation problem for three(3) pure water factories in Nigeria from different states.
\end{enumerate}
	
\newpage

\begin{center}
\subsection{Significance of the Project}
\end{center}

The significance of this project is to use the methods of transportation problem in mathematical operation research to know the best ways to transport goods and also record the lowest cases of loss.

Companies have to decide on when and how they will transport their goods and the outcome of their choices can result in transportation accidents or loss of money. 

If companies are aware of the best possible ways to transport goods to places, they will be able to record the lowest number of accidents when travelling and also make the highest amount of profit.

\newpage

\begin{center}
	\section{LITERATURE REVIEW}
\end{center}
\subsection{Literature Review}

https://jespublication.com/upload/2020-202003107.pdf

ABSTRACT

The transportation of rice from different location to different locations was examined and the problem was formulated as a LPP model.

We obtained an IBFS to the problem by NORTH-WEST
CORNER METHOD (NWCM) and EASY
QUICK METHOD (EQM) and compared
the results and displayed in the tables. The
key idea in EQM is to minimize the best
combinations of the solution to reach the
optimal solution. This obtained the
best initial feasible solution to the
transportation problem and performs faster
than the existing methods with a minimal
computation time and less complexity.

The proposed method is an attractive
alternative to traditional problem solution
methods. MODI method has applied for optimal solution for
the effectiveness of the proposed method.

INTRODUCTION
Industries require planning in transportation with small transporting cost to maximize profit. Transportation problem can be used in many fields such as scheduling, personnel assignment, product mix problems. In the solution procedure of transportation problem, finding an Initial Basic Feasible Solution (IBFS) is the prerequisite to obtain the optimal solution.

This research aims to propose an algorithm
“EQM” to obtain an IBFS for the transportation problems. Obtained results show that the proposed algorithm is effective in solving transportation problems.

Transportation model provides a greater impact on the
transportation of the commodities from the manufacturing places. The basic Transportation problem was initially
proposed by Hitch Cock. A day’s transportation problem has become a standard application for industrial organizations having several manufacturing units, warehouses and distribution centers.

A Transportation problem is one of the
earliest and most important applications of
LPP. Description of a classical
transportation problem can be given as
follows. A certain amount of homogeneous
commodity is available at number of
sources/origins and a fixed amount is
required to meet the demand at each number
of destinations/distribution centers. Then
finding an optimal schedule of shipment of
the commodity with the satisfaction of demands at each destination is the main goal
of the problem. In 1941, Hitchcock [1]
developed the basic transportation problem
along with the constructive method of
solution and later in 1949 Koopmans [2]
discussed the problem in detail. Again in
1951 Dantzig [3] formulated the
transportation problem as LPP and also
provided the solution method. The
transportation problem in which the
objective is to minimize the total cost of
shipping a single commodity from a number
of sources (m) to a number of destinations or
sinks (n). Because of the special structure of
the transportation problem, a special
algorithm can be designed to find an optimal
solution efficiently. The most important and
successful applications in the optimization
refers to transportation problem (TP).


\newpage

\begin{center}
	\section{METHODOLOGY}
\end{center}

This 

\newpage

\begin{center}
	\section{TRANSPORTATION PROBLEM}
\end{center}

\newpage

\begin{center}
	\section{SUMMARY}
\end{center}
\subsection{Summary}
\subsection{Conclusion}

\newpage

\begin{center}
	\section{REFERENCE}
\end{center}

\end{document}