\documentclass{article}
% Comment the following line to NOT allow the usage of umlauts
\usepackage{fullpage}
\usepackage[utf8]{inputenc}
\usepackage{amsmath}
\usepackage{amsfonts}
\usepackage{amssymb}
\renewcommand{\baselinestretch}{1.5}
\newcommand\tab[1][1cm]{\hspace*{1}}
\pagenumbering{arabic}
\usepackage{comment }
\usepackage{multicol}
\usepackage{geometry}
%\usepackage{sectsty}
%\usepackage{scrextend}
%\changefontsizes[40pt]{25pt}
%\usepackage[english]{babel}


\begin{document}
	\begin{center}
	\Huge\textbf{Ajibola Segun Emmanuel} \\
	\vspace{5mm} %5mm vertical space
	\Huge\textbf{180551012} \\
	\vspace{5mm} %5mm vertical space
	\Huge\textbf{B.Sc Mathematics} \\ 
	\end{center}
	\vspace{50mm} %20mm vertical space
	\begin{center}
	\Large\textbf{PROJECT TOPIC:} \\
	\vspace{5mm} %5mm vertical space
	\Huge\textbf{\Huge OPTIMIZATION FOR TRANSPORTATION PROBLEMS FOR THREE NIGERIAN TRANSPORTING COMPANIES}\par
	\end{center}

\newpage
\begin{center}
	\subsection{Certification}
\end{center}

I certify that this work by AJIBOLA SEGUN EMMANUEL with matriculation number 180551012 under my supervision in the Department of Mathematics, Lagos State University, Ojo, Lagos.

\vspace{30mm}

\begin{multicols}{2}

\noindent
Mr Oni \\ 
(Supervisor) \\ 
Department of Mathematics \\ 
Lagos State University \\ 
Ojo, Lagos. \\ 

\noindent
..................... \\
Signature \\
..................... \\
Date \\

\noindent
Dr Oyebo Y. T. \\
Ag. Head \\
Department of Mathematics \\
Lagos State University \\
Ojo, Lagos. \\

\noindent
..................... \\
Signature \\
..................... \\
Date \\

\end{multicols}

\newpage

\begin{center}
	\subsection{Dedication}
\end{center}


I dedicate this seminar work to God Almighty for his protection over my course of study, and also to my family, the family of Mr and Mrs Ajibola, my relatives and friends. Thanks for all the support.

\newpage

\begin{center}
	\subsection{Acknowledgement}
\end{center}

I am grateful to God for giving me the ability to work on this project. I also thank my supervisor, Mr Oni, for his guidance along this project work.

My gratitude also goes to my parents, relatives, course mates, and friends for their sincere love and support during my time in school. May God bless you all.

\newpage

\begin{center}
	\subsection{Abstract}
\end{center}


Operation Research, also shorten as OR, is a field that deals with the development and application of analytical methods to improve decision making. 

It is used to solve complex problems ranging from business to art through mathematical analysis. Its technique can be applied in various fields of agriculture, transportation and business management.

Operation reaserch's method to optimize conditions can affect humans decision and make them make life-saving decisions.

\newpage

\subsection{Table of Contents}
\tableofcontents

\newpage

\begin{center}
\section{INTRODUCTION}
\textbf{CHAPTER 1}
\end{center}
Bjarke Bundgaard Ingels, a Danish architect, founder and creative partner of Bjarke Ingels Group said "In Copenhagen, there's a long-term commitment to creating a well-functioning pedestrian city where all forms of movement - pedestrian, bicycles, cars, public transportation - are accommodated with equal priority". As at 2022, Elon Musk, the richest man on earth said "If we drive down the cost of transportation in space, we can do great things". This two great quotes from experienced people in different fields shows the importance of transportation in our daily lives and businesses.

During the administration of President Olusegun Obasanjo in Nigeria, the National Poverty Eradication Programme (NAPEP) program was created in 2001 by the Nigerian government aiming to reduce poverty and provide economic empowerment to the people of the country. This prompted the high use of the popular tricyle known as Keke Maruwa and it was soon popularly known as Keke Napep across the nation. This notion was popular and can be attributed to the widespread National Poverty Eradication Programme NAPEP, which was inaugurated under the current government.

Without any doubt, this short story clearly shows that transportation is one of most important functions in any country, developing or developed countries. It is one of the core activity necessary for the advancement of human life and the growth in industrial sector.

In Nigeria, the length of distance travelled is over two hundred thousand kilometres in road, rail, water ways, commercial harbours and airports sector.

\newpage

\begin{center}
\subsection{Statement of Problem}
\end{center}
With the high kilometres travelled in Nigeria, this brings risks and cost problems for industries sending goods from one place in and out of the country.

As good and important the transport system of a state can be, it's disadvantages poses a high cost. Millions can be lost in a day and in most bad cases, human lives can be lost.

Poor transport system, design and decisions increases the damages in movement of materials from one place to another to conduct business activities.

According to Logistics Update Africa, Nigeria's Federal Road Safety Corps disclosed that Nigeria lost N39 billion to trailer and tanker crashes in 2018, involving about 650 vehicles.

\newpage

\begin{center}
	\subsection{Scope of Study}
\end{center}
This project will highlight the transportation problem in the sector. It will use an operation research called transportation problem to analyse a transport system and enumerates ways that cost of bad transportation problems can be reduced, thus increasing income.

Linear programming will be used in this project, the type of the programming used is called transportation problem. It is a type of linear programming where the objective helps to minimize transportation cost of goods from a place to different locations.

Three (3) pure water factories in Nigeria will be used as examples, the transportation details will be stated and the transportation optimisation system will be used to reduce cost of transportation to get the maximum profits.

\newpage

\begin{center}
\subsection{Objectives of Study}
\end{center}

The objectives of this project is stated below: \newline
\begin{enumerate}
\item Enumerate the importance of optimized transport system
\item Review the Transportation Problem of Linear Programming
\item Use the application of Linear Programming to optimize the transportation problem for three(3) pure water factories in Nigeria from different states.
\end{enumerate}
	
\newpage

\begin{center}
\subsection{Significance of the Project}
\end{center}

The significance of this project is to use the methods of transportation problem in mathematical operation research to know the best ways to transport goods and also record the lowest cases of loss.

Companies have to decide on when and how they will transport their goods and the outcome of their choices can result in transportation accidents or loss of money. 

If companies are aware of the best possible ways to transport goods to places, they will be able to record the lowest number of accidents when travelling and also make the highest amount of profit.

\newpage

\begin{center}
	\section{LITERATURE REVIEW}
	\textbf{CHAPTER 2}
\end{center}
\subsection{Literature Review}

\noindent ABSTRACT \break
The transportation of rice from different location to different locations was examined and the problem was formulated as a LPP model.

We obtained an IBFS to the problem by NORTH-WEST
CORNER METHOD (NWCM) and EASY QUICK METHOD (EQM) and compared the results and displayed in the tables. The key idea in EQM is to minimize the best combinations of the solution to reach the optimal solution. This obtained the best initial feasible solution to the transportation problem and performs faster than the existing methods with a minimal computation time and less complexity.

The proposed method is an attractive alternative to traditional problem solution methods. MODI method has applied for optimal solution for the effectiveness of the proposed method. \break

\noindent INTRODUCTION \break
Industries require planning in transportation with small transporting cost to maximize profit. Transportation problem can be used in many fields such as scheduling, personnel assignment, product mix problems. In the solution procedure of transportation problem, finding an Initial Basic Feasible Solution (IBFS) is the prerequisite to obtain the optimal solution.

This research aims to propose an algorithm “EQM” to obtain an IBFS for the transportation problems. Obtained results show that the proposed algorithm is effective in solving transportation problems. Transportation model provides a greater impact on the
transportation of the commodities from the manufacturing places. The basic Transportation problem was initially proposed by Hitch Cock. A day’s transportation problem has become a standard application for industrial organizations having several manufacturing units, warehouses and distribution centers.

Description of a classical transportation problem can be given as follows. A certain amount of homogeneous
commodity is available at number of sources/origins and a fixed amount is required to meet the demand at each number of destinations/distribution centers. Then finding an optimal schedule of shipment of the commodity with the satisfaction of demands at each destination is the main goal of the problem. The
transportation problem in which the objective is to minimize the total cost of shipping a single commodity from a number of sources (m) to a number of destinations or sinks (n). The most important and successful applications in the optimization refers to transportation problem (TP).

The basic steps for obtaining an optimum solution to a transportation problem are: (P.K. Gupta and Man Mohan, (2003)`.

Step 1: Mathematical model of the problem
Step 2: Finding an Initial Basic Feasible Solution (IBFS)
Step 3: To test whether the solution is an
optimal one or not. If not, improve it further till the optimality is achieved. 

Most of the time the initial basic feasible solution of transportation problem is calculated by using the methods of NorthWest Corner Method or Least-Cost Method or Vogel’s Approximation Method, and then finally the optimality is checked by MODI (Modified Distribution Method).

In this paper, we considered a transportation problem of an essential item, rice, from the different origins to different destinations and formulated the problem as a LPP model. We obtained an IBFS to the problem by NWCM
and EQM and compared the results and displayed in the tables. The key idea in EQM is to minimize the combinations of the solution by choosing the best least cells to reach the optimal solution. Comparatively,
applying the EQM in the proposed method obtains the best Initial Basic Feasible Solution to a transportation problem and performs faster than the existing methods with a minimal computation time and less complexity.

There are several different algorithms to solve transportation problem that represented as LP model like the algebraic procedures of the simplex method. The standard scenario for solving transportation problems is working by sending units of a product across
a network of highways that connect a given set of cities. Each city is considered as a source in that units will be shipped out from, while units are demanded there when the city is considered as a sink. In this scenario, each sink has a given demand, the
source has a given supply, and the airway that connects source with sink as a pair has a given transportation cost/(shipment unit). The problem is to determine an optimal transportation scheme that is to minimize the
total of the shipments cost between the nodes in the network model, subject to supply and demand constraints. As well as, this structure arises in many applications such as; the sources represent warehouses
and the sinks represent retail outlets. Moreover, Ad-hoc networks are designed dynamically by group of mobile devices. In Ad-hoc network, nodes between source and destination act as a routers so that source node can communicate with the destination node.

North-West Corner method (NWCM) is one of the conventional methods that give better Initial Basic Feasible Solution (IBFS) of a Transportation Problem (TP). This method is very effective as it provides step by step solution and it is very simple to find IBFS through this method.

Initial Basic Feasiblt Solution using easy quick method (EQM) is done by checking if the matrix balanced or not, if the total supply is equal to the total demand, then the matrix is balanced. If the total supply is not equal to the total demand, then we add a dummy row or column as needed to make supply is equal to the demand.

In moving towards optimality, to verify whether the above IBFS using NQM is optimal solution or not, we can apply MODI (Modified Distribution method) or u-v method. \break

\noindent CONCLUSION \break
In this study, we proposed EQM for finding the IBFS to the transportation problem. It refers to choose the best distribution of cost and time from the all combinations. The EQM obtained the optimal solution or the closest to optimal solution with a minimum computation time. As well as, use of EQM reduces the complexity of the problems.

\newpage

\begin{center}
	\section{METHODOLOGY}
	\textbf{CHAPTER 3}
\end{center}

This chapter explains the methodology for optimizing the transportation problem of the three pure water companies, x, x and x.
The aim of the solution is to reduce the transportation costs by finding solutions to minimize costs of transporting goods from one location to another location.

In the next paragrapgh, we will look at the foundation of Transportation problem, what it is and how it works.

/// m origins (each with a supply si) to n destinations (each with a demand dj), when the unit shipping cost from an origin, i, to a destination, j is Cij./ ///

3.2 BASICS OF TRANSPORTATION PROBLEM

Transportation problem is a special kind of Linear Programming Problem also known as LPP, goods that are transported from a set of locations to another set of destinations subjected to the supply and demand of the sources respectively and this is to minimize the total cost of transportation.

Finding solution to transportation problems has been crucial to various disciplines. An Initial Basic Feasible Solution(IBFS) for the transportation problem will be obtained by using one of the methods of solving transportation problems. The best optimal condition will be selected.

Transportation uses different method of solving it's problem, they are North-West corner rule, Minimum Cost Method and Vogel’s Approximation Method.

The idea is to get all the data about the goods leaving a location or origin and using that data to optimize the cost of transporting it to another destination.

In this project work, the first location will be the area of production (origins) and the second location(destinations).

3.3 MATHEMATICAL MODELLING

A pure water company production's location is denoted by (m). Each bags of pure water will be transported or supplied to different retail shops depending on the retail shop's demand. The retail shops is denoted by (n).

Transportation model deal with getting the minimum cost plan to transport bags of pure water from a number of production places(m) to number of destination(n).

THE DESICION VARIABLES

- m = Number of sources (i = 1 … m)

- n = Number of destinations (j = 1 … n)

- Si = The number of supply unit required at source i, (i=1, 2, 3....... a).

- Dj = The number of demand unit required at destination j, (j=1, 2,3..... b)

- Cij = The transportation unit cost for transporting the units from sources i to destination j.

- Xij = The number of pure water bags transported from i to j.

- Cij = The transportation cost for transporting bags of pure water from sources i to destination j.

To determine the optimal number of units or goods that can be transported from the sources i to destination j, we get the objective function using linear programming principle.

THE OBJECTIVE FUNCTION

The objective function is the target variable. It is the cost function or the total amount spent for transporting the goods from the source to the destination. The idea is to use the function to minimize the cost of transporting while also satisfying all the supply and demand restrictions.

The objective function is gotten by the sum product of the cost per unit per km and the decision variables, the total cost is directly proportional to the sum product of the number of units shipped and cost of transport per unit per Km.

The objective function to minimize is denoted by M in this project work.

 \[M = \sum_{i=1}^{m} \sum_{j=1}^{n} Cij Xij \]
 
Subject to

 \[M = \sum_{j=1}^{n} Xij = Sij. For  i = 1, 2, ...m \]

 \[M = \sum_{i=1}^{m} Xij = Dij. For j = 1, 2, ...n\]

/// input later///
A transportation problem said to be balanced if the
supply from all sources equals the total demand in all
destinations.

Otherwise it is called unbalanced.
A transportation problem is said to be balanced
if the total supply from all sources equals the total
demand in all destinations
Otherwise it is called unbalanced
///     ////

3. THE CONSTRAINTS

The constraints means that you must meet some the required conditions or demand and not exceed supply at each supply center.

The constraints are formulated concerning the demand and supply.. The importance of constraints is to ensure they the solution satisfy all the supply and demand restrictions.

3. TYPES OF TRANSPORTAION PROBLEM

There are two  types of transportation problems based on the problem:

1. Balanced Transportation Problems: The Transportation problem is balanced when the total supply is equal to the total demand. For example, if a retail outlet demands for 50 bags of purewater, and 50 bags was transported, the optimization method used will be balanced.

\[\sum_{i=1}^{m} Ai = \sum_{j=1}^{n}Bj \]


2. Unbalanced Transportation Problems: The Transportation problem is not balanced when the total supply is not equal to the total demand. For example, 50 bags of water is demanded and 40 or 60 bags is supplied.

\[\sum_{i=1}^{m} Ai \neq \sum_{j=1}^{n}Bj \]

\[\sum_{i=1}^{m} Ai > \sum_{j=1}^{n}Bj \]

\[\sum_{i=1}^{m} Ai < \sum_{j=1}^{n}Bj \]

When the supply is higher than the demand, a dummy destination is introduced in the equation to make it equal to the supply (with 0 shipping costs) and when the demand is higher than the supply, a dummy source is introduced in the equation to make it equal to the demand.

To proceed with the solution of transportation problems, the first step is to check if it is balanced or not.


3.5 DEGENCY IN TRANSPORTATION PROBLEM 
Degeneracy exists in a transportation problem when the number of filled cells is less than the number of rows plus the number of columns minus one (m + n - 1). Degeneracy may be observed either during the initial allocation when the first entry in a row or column satisfies both the row and column requirements or during the Stepping stone method application, when the added and subtracted values are equal.
Transportation with m-origins and n-destinations can have m+n-1 positive basic variables, otherwise the basic solution degenerates. So whenever the number of basic cells is less than m + n-1, the transportation problem is degenerate. To resolve the degeneracy, the positive variables are augmented by as many zero-valued variables as is necessary to complete m +n –1 basic variable. 


3.6 THE INITIAL BASIC FEASIBLE SOLUTION (BFS) 
Let us consider a T.P involving m origins and n destinations. Since the sum of origin capacities equals the sum of destination requirements, a feasible solution always exists. Any feasible solution satisfying m + n – 1 of the m + n constraints is a redundant one and hence can be deleted. This also means that a feasible solution to a T.P can have at the most only m + n – 1 strictly positive component, otherwise the solution will degenerate. It is always possible to assign an initial feasible solution to a T.P. in such a manner that the rim requirements are satisfied. This can be achieved either by inspection or by following some simple rules. We begin by imagining that the transportation table is blank i.e. initially all Xij = 0. The simplest procedures for initial allocation is discussed in the following section below:
Feasible Solution (F.S):  A set of non-negative allocations xij > 0 which satisfies the row and column restrictions is known as feasible solution. 
Basic Feasible Solution (B.F.S): A feasible solution to a m-origin and n-destination problem is said to be basic feasible solution if the number of positive allocations are (m+n–1). But if the number of allocations in a basic feasible solutions are less than (m+n–1), it is called degenerate basic feasible solution (DBFS) (Otherwise non-degenerate). 
Optimal Solution: A feasible solution (not necessarily basic) is said to be optimal if it minimizes the total transportation cost. 
Cell: It is a small compartment in the transportation tableau. 
Circuit: A circuit is a sequence of cells (in the balanced transportation tableau) such that 
It starts and ends with the same cell. 
Each cell in the sequence can be connected to the next member by a horizontal or vertical line in the tableau. 
Allocation: The number of units of items transported from a source to a destination which is recorded in a cell in the transportation tableau. 
Basic Variables: The variables in a basic solution whose values are obtained as the simultaneous solution of the system of equations that comprise the functional constraints

3.6 METHODS OF SOLVING TRANSPORTATION PROBLEM 
Transportation models do not start at the origin where all decision values are zero; they must instead be given an initial feasible solution The solution algorithm to a transpiration problem can be summarized into following steps: 
Step 1. Formulate the problem and set up in the matrix form, the formulation of transportation problem is similar to LP problem formulation. Here the objective function is the total transportation cost and the constraints are the supply and demand available at each source and destination respectively. 
Step 2. Obtain an initial basic feasible solution. This initial basic solution can be obtained by using any of the following methods: 
North West Corner Method 
Least Cost Cell Method 
Vogel Approximation Method  (VAM)
The solution obtained by any of the above methods must fulfil the following conditions: 
The solution must be feasible, i.e., it must satisfy all the supply and demand constraints. This is called RIM CONDITION. 
ii. The number of positive allocation must be equal to m + n – 1, where, m is number of rows and n is number of columns
The solution that satisfies the above mentioned conditions are called a non-degenerate basic feasible solution. 
Step 3. Test the initial solution for optimality. Using any of the following methods can test the optimality of obtained initial basic solution: 
Stepping Stone Method 
ii. Modified Distribution Method (MODI).
If the solution is optimal then stop, otherwise, determine a new improved solution. 
Step 4. Updating the solution (Repeat Step 3 until the optimal solution is arrived at).



\newpage

\begin{center}
	\section{TRANSPORTATION PROBLEM}
\end{center}

\newpage

\begin{center}
	\section{SUMMARY}
\end{center}
\subsection{Summary}
\subsection{Conclusion}

\newpage

\begin{center}
	\section{REFERENCE}
\end{center}

\end{document}